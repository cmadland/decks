% Options for packages loaded elsewhere
\PassOptionsToPackage{unicode}{hyperref}
\PassOptionsToPackage{hyphens}{url}
%
\documentclass[
]{book}
\usepackage{amsmath,amssymb}
\usepackage{lmodern}
\usepackage{iftex}
\ifPDFTeX
  \usepackage[T1]{fontenc}
  \usepackage[utf8]{inputenc}
  \usepackage{textcomp} % provide euro and other symbols
\else % if luatex or xetex
  \usepackage{unicode-math}
  \defaultfontfeatures{Scale=MatchLowercase}
  \defaultfontfeatures[\rmfamily]{Ligatures=TeX,Scale=1}
\fi
% Use upquote if available, for straight quotes in verbatim environments
\IfFileExists{upquote.sty}{\usepackage{upquote}}{}
\IfFileExists{microtype.sty}{% use microtype if available
  \usepackage[]{microtype}
  \UseMicrotypeSet[protrusion]{basicmath} % disable protrusion for tt fonts
}{}
\makeatletter
\@ifundefined{KOMAClassName}{% if non-KOMA class
  \IfFileExists{parskip.sty}{%
    \usepackage{parskip}
  }{% else
    \setlength{\parindent}{0pt}
    \setlength{\parskip}{6pt plus 2pt minus 1pt}}
}{% if KOMA class
  \KOMAoptions{parskip=half}}
\makeatother
\usepackage{xcolor}
\IfFileExists{xurl.sty}{\usepackage{xurl}}{} % add URL line breaks if available
\IfFileExists{bookmark.sty}{\usepackage{bookmark}}{\usepackage{hyperref}}
\hypersetup{
  pdftitle={Presentations by Colin Madland},
  pdfauthor={Colin Madland},
  hidelinks,
  pdfcreator={LaTeX via pandoc}}
\urlstyle{same} % disable monospaced font for URLs
\usepackage{longtable,booktabs,array}
\usepackage{calc} % for calculating minipage widths
% Correct order of tables after \paragraph or \subparagraph
\usepackage{etoolbox}
\makeatletter
\patchcmd\longtable{\par}{\if@noskipsec\mbox{}\fi\par}{}{}
\makeatother
% Allow footnotes in longtable head/foot
\IfFileExists{footnotehyper.sty}{\usepackage{footnotehyper}}{\usepackage{footnote}}
\makesavenoteenv{longtable}
\usepackage{graphicx}
\makeatletter
\def\maxwidth{\ifdim\Gin@nat@width>\linewidth\linewidth\else\Gin@nat@width\fi}
\def\maxheight{\ifdim\Gin@nat@height>\textheight\textheight\else\Gin@nat@height\fi}
\makeatother
% Scale images if necessary, so that they will not overflow the page
% margins by default, and it is still possible to overwrite the defaults
% using explicit options in \includegraphics[width, height, ...]{}
\setkeys{Gin}{width=\maxwidth,height=\maxheight,keepaspectratio}
% Set default figure placement to htbp
\makeatletter
\def\fps@figure{htbp}
\makeatother
\setlength{\emergencystretch}{3em} % prevent overfull lines
\providecommand{\tightlist}{%
  \setlength{\itemsep}{0pt}\setlength{\parskip}{0pt}}
\setcounter{secnumdepth}{5}
\usepackage{booktabs}
\usepackage{amsthm}
\makeatletter
\def\thm@space@setup{%
  \thm@preskip=8pt plus 2pt minus 4pt
  \thm@postskip=\thm@preskip
}
\makeatother
\ifLuaTeX
  \usepackage{selnolig}  % disable illegal ligatures
\fi
\usepackage[]{natbib}
\bibliographystyle{apalike}

\title{Presentations by Colin Madland}
\author{Colin Madland}
\date{Last updated: 2022-05-16}

\begin{document}
\maketitle

{
\setcounter{tocdepth}{1}
\tableofcontents
}
\hypertarget{welcome}{%
\chapter*{Welcome}\label{welcome}}
\addcontentsline{toc}{chapter}{Welcome}

Please use the table of contents on the left to navigate through my presentations.

\hypertarget{otessa22---assessment-and-digital-technology-in-higher-education}{%
\chapter*{OTESSA22 - Assessment and Digital Technology in Higher Education}\label{otessa22---assessment-and-digital-technology-in-higher-education}}
\addcontentsline{toc}{chapter}{OTESSA22 - Assessment and Digital Technology in Higher Education}

\hypertarget{introduction}{%
\section*{Introduction}\label{introduction}}
\addcontentsline{toc}{section}{Introduction}

\hypertarget{colin-madland-phd-candidate-university-of-victoria}{%
\subsection*{Colin Madland, PhD Candidate, University of Victoria}\label{colin-madland-phd-candidate-university-of-victoria}}
\addcontentsline{toc}{subsection}{Colin Madland, PhD Candidate, University of Victoria}

\textbf{Presented Online at OTESSA22, May 17, 2022}

\begin{quote}
I acknowledge that the land where I currently live and work remains the traditional, ancestral, and unceded land of the \texttt{syilx} (silks) people, whose historical stewardship of and connections to the land continue to today. I am grateful to be an uninvited guest on this land. \href{https://wfn.ca}{To learn more, please visit the Westbank First Nation website.}
\end{quote}

\begin{figure}
\centering
\includegraphics{assets/otessa22/kalamoir.jpg}
\caption{Figure 1. Author's bicycle overlooking Okanagan Lake.}
\end{figure}

\hypertarget{background}{%
\subsection*{Background}\label{background}}
\addcontentsline{toc}{subsection}{Background}

\hypertarget{scriven-1967}{%
\subsubsection*{Scriven, 1967}\label{scriven-1967}}
\addcontentsline{toc}{subsubsection}{Scriven, 1967}

\begin{quote}
Scriven, M. (1967). \emph{The methodology of evaluation.} In B. O. Smith (Ed.), \emph{Perspectives of curriculum evaluation}. Rand McNally
\end{quote}

\begin{itemize}
\tightlist
\item
  distinction between \texttt{formative} and \texttt{summative}
\end{itemize}

\hypertarget{bloom-1968}{%
\subsubsection*{Bloom, 1968}\label{bloom-1968}}
\addcontentsline{toc}{subsubsection}{Bloom, 1968}

\begin{quote}
Bloom, B. (1968). Learning for Mastery. Instruction and Curriculum. Regional Education Laboratory for the Carolinas and Virginia, Topical Papers and Reprints, Number 1. \emph{Evaluation Comment, 1}(2), 12.
\end{quote}

\begin{itemize}
\tightlist
\item
  Incorporated \texttt{formative} and \texttt{summative} distinction into his ideas about \texttt{mastery\ learning}
\end{itemize}

\hypertarget{mislevy-1994}{%
\subsubsection*{Mislevy, 1994}\label{mislevy-1994}}
\addcontentsline{toc}{subsubsection}{Mislevy, 1994}

\begin{quote}
Mislevy, R. J. (1994). Test theory reconcieved. \emph{ETS Research Report Series, 1994}(1), i--38. \url{https://doi.org/10/gjm236}
\end{quote}

\begin{itemize}
\item
  \begin{quote}
  test theory is machinery for reasoning from students' behavior to conjectures about their competence, as framed in a particular conception of competence.''(p.~4).
  \end{quote}
\end{itemize}

\hypertarget{black-and-wiliam-1998}{%
\subsubsection*{Black and Wiliam, 1998}\label{black-and-wiliam-1998}}
\addcontentsline{toc}{subsubsection}{Black and Wiliam, 1998}

\begin{quote}
Black, P., \& Wiliam, D. (1998). Assessment and Classroom Learning. \emph{Assessment in Education: Principles, Policy \& Practice, 5}(1), 7--74. \url{https://doi.org/10/fpnss4}
\end{quote}

\begin{itemize}
\item
  major review of the literature on \texttt{formative\ assessment}\\
\item
  describe formative assessment as encouraging gains in achievement that were

\begin{verbatim}
> among the largest ever reported for educational interventions (p. 61)
\end{verbatim}
\end{itemize}

\hypertarget{pellegrino-et-al.-2001}{%
\subsubsection*{Pellegrino et al., 2001}\label{pellegrino-et-al.-2001}}
\addcontentsline{toc}{subsubsection}{Pellegrino et al., 2001}

\begin{quote}
Pellegrino, J. W., Chudowsky, N., \& Glaser, R. (2001). \emph{Knowing What Students Know: The Science and Design of Educational Assessment}. National Academies Press. \url{https://doi.org/10.17226/10019}
\end{quote}

\begin{itemize}
\tightlist
\item
  ``a process of drawing reasonable inferences about what students know on the basis of evidence derived from observations of what they say, do, or make in selected situations'' (p.~112)\\
\item
  ``reasoning from evidence'' (p.~43)
\end{itemize}

\hypertarget{assessment-triangle}{%
\paragraph*{Assessment Triangle}\label{assessment-triangle}}
\addcontentsline{toc}{paragraph}{Assessment Triangle}

\begin{figure}
\centering
\includegraphics{assets/otessa22/assessment-triangle.png}
\caption{Figure 2. Assessment Triangle from Pellegrino et al.~(2001)}
\end{figure}

\hypertarget{cognition}{%
\subparagraph*{Cognition}\label{cognition}}
\addcontentsline{toc}{subparagraph}{Cognition}

\begin{itemize}
\tightlist
\item
  a cognitive model of the domain
\end{itemize}

\hypertarget{observation}{%
\subparagraph*{Observation}\label{observation}}
\addcontentsline{toc}{subparagraph}{Observation}

\begin{itemize}
\tightlist
\item
  a performance task used to gather data regarding learner achievement
\end{itemize}

\hypertarget{interpretation}{%
\subparagraph*{Interpretation}\label{interpretation}}
\addcontentsline{toc}{subparagraph}{Interpretation}

\begin{itemize}
\tightlist
\item
  an inference or judgement of the learner's achievement in relation to the model of the domain
\end{itemize}

\hypertarget{approaches-to-learning}{%
\subsection*{Approaches to Learning}\label{approaches-to-learning}}
\addcontentsline{toc}{subsection}{Approaches to Learning}

\hypertarget{biggs-1993}{%
\subsubsection*{Biggs, 1993}\label{biggs-1993}}
\addcontentsline{toc}{subsubsection}{Biggs, 1993}

\begin{quote}
Biggs, J. B. (1993). From Theory to Practice: A Cognitive Systems Approach. \emph{Higher Education Research \& Development, 12}(1), 73--85. \url{https://doi.org/10/ccdmd9}
\end{quote}

\begin{figure}
\centering
\includegraphics{assets/otessa22/3p-model.png}
\caption{Figure 3. 3-P Model of Teaching and Learning adapted from Biggs (1993)}
\end{figure}

\hypertarget{presage}{%
\paragraph*{Presage}\label{presage}}
\addcontentsline{toc}{paragraph}{Presage}

\begin{itemize}
\tightlist
\item
  factors that precede learning activities

  \begin{itemize}
  \tightlist
  \item
    learner factors

    \begin{itemize}
    \tightlist
    \item
      prior knowledge\\
    \item
      educational experience\\
    \item
      affective states\\
    \item
      wellness (physical \& mental)\\
    \end{itemize}
  \item
    teacher factors

    \begin{itemize}
    \tightlist
    \item
      vertical \& horizontal discourses (Bernstein, 1999)\\
    \item
      institutional policies\\
    \item
      department norms\\
    \item
      educational experiences
    \end{itemize}
  \end{itemize}
\end{itemize}

\hypertarget{process}{%
\paragraph*{Process}\label{process}}
\addcontentsline{toc}{paragraph}{Process}

\begin{itemize}
\tightlist
\item
  learning focused activities

  \begin{itemize}
  \tightlist
  \item
    reading, writing, discussing, building, creating, synthesizing, researching, sharing, debating, publishing\ldots{}
  \end{itemize}
\item
  surface approaches

  \begin{itemize}
  \tightlist
  \item
    using low-level cognitive skills when high-level cognitive skills are required
  \end{itemize}
\item
  deep approaches

  \begin{itemize}
  \tightlist
  \item
    using high-level cognitive skills for tasks which require them
  \end{itemize}
\end{itemize}

\hypertarget{product}{%
\paragraph*{Product}\label{product}}
\addcontentsline{toc}{paragraph}{Product}

\begin{itemize}
\tightlist
\item
  learner achievement of outcomes (intended or emergent)
\item
  fed back into the system

  \begin{itemize}
  \tightlist
  \item
    informs learners and instructors
  \end{itemize}
\end{itemize}

\hypertarget{conceptions-of-assessment}{%
\subsection*{Conceptions of Assessment}\label{conceptions-of-assessment}}
\addcontentsline{toc}{subsection}{Conceptions of Assessment}

\hypertarget{brown-1994-1996}{%
\subsubsection*{Brown, 1994; 1996}\label{brown-1994-1996}}
\addcontentsline{toc}{subsubsection}{Brown, 1994; 1996}

\begin{quote}
Brown, G. T. L. (2004). Teachers' conceptions of assessment: Implications for policy and professional development. \emph{Assessment in Education: Principles, Policy \& Practice, 11}(3), 301--318. \url{https://doi.org/10.1080/0969594042000304609}
\end{quote}

\begin{quote}
Brown, G. T. L. (2006). Teachers' Conceptions of Assessment: Validation of an Abridged Version. \emph{Psychological Reports, 99}(1), 166--170. \url{https://doi.org/10/bf67hf}
\end{quote}

\begin{itemize}
\tightlist
\item
  general mental structure, encompassing beliefs, meanings, concepts, propositions, rules, mental images, preferences

  \begin{itemize}
  \tightlist
  \item
    improvement of teaching and learning,\\
  \item
    school accountability,\\
  \item
    student accountability, or\\
  \item
    treating assessment as irrelevant.
  \end{itemize}
\end{itemize}

\hypertarget{fletcher-et-al.-2012}{%
\subsubsection*{Fletcher et al., 2012}\label{fletcher-et-al.-2012}}
\addcontentsline{toc}{subsubsection}{Fletcher et al., 2012}

\begin{quote}
Fletcher, R. B., Meyer, L. H., Anderson, H., Johnston, P., \& Rees, M. (2012). Faculty and Students Conceptions of Assessment in Higher Education. \emph{Higher Education, 64}(1), 119--133. \url{https://doi.org/10/ctccpq}
\end{quote}

\begin{itemize}
\tightlist
\item
  instructors were more likely than learners to view assessment as consistent and trustworthy methods to understand and improve learning\\
\item
  learners were more likely to have negative views of assessment and viewed it as a measure of student and institutional accountability.
\end{itemize}

\hypertarget{earl-2013}{%
\subsubsection*{Earl, 2013}\label{earl-2013}}
\addcontentsline{toc}{subsubsection}{Earl, 2013}

\begin{quote}
Earl, L. M. (2013). \emph{Assessment as learning: Using classroom assessment to maximize student learning (Second edition)}. Corwin Press.
\end{quote}

\begin{itemize}
\tightlist
\item
  Assessment \emph{OF} Learning

  \begin{itemize}
  \tightlist
  \item
    summative
  \end{itemize}
\item
  Assessment \emph{FOR} Learning

  \begin{itemize}
  \tightlist
  \item
    formative
  \end{itemize}
\item
  Assessment \emph{AS} Learning

  \begin{itemize}
  \tightlist
  \item
    metacognitive
  \end{itemize}
\end{itemize}

\hypertarget{approaches-to-assessment}{%
\subsection*{Approaches to Assessment}\label{approaches-to-assessment}}
\addcontentsline{toc}{subsection}{Approaches to Assessment}

\hypertarget{assessment-in-higher-education}{%
\section*{Assessment in Higher Education}\label{assessment-in-higher-education}}
\addcontentsline{toc}{section}{Assessment in Higher Education}

bearmanSupportAssessmentPractice2016
\citet{lipnevichWhatGradesMean2020}; \citet{masseyAssessmentLiteracyCollege2020}
\#\# Technology-Mediated Assessment \{-\}

\hypertarget{research-directions}{%
\section*{Research Directions}\label{research-directions}}
\addcontentsline{toc}{section}{Research Directions}

\hypertarget{references}{%
\section*{References}\label{references}}
\addcontentsline{toc}{section}{References}

Bernstein, B. (1999). Vertical and Horizontal Discourse: An Essay. \emph{British Journal of Sociology of Education, 20}(2), 157--173. JSTOR. \url{https://doi.org/10/ftmsvc}

Biggs, J. B. (1993). From Theory to Practice: A Cognitive Systems Approach. \emph{Higher Education Research \& Development, 12}(1), 73--85. \url{https://doi.org/10/ccdmd9}

Black, P., \& Wiliam, D. (1998). Assessment and Classroom Learning. \emph{Assessment in Education: Principles, Policy \& Practice, 5}(1), 7--74. \url{https://doi.org/10/fpnss4}

Bloom, B. (1968). Learning for Mastery. Instruction and Curriculum. Regional Education Laboratory for the Carolinas and Virginia, Topical Papers and Reprints, Number 1. \emph{Evaluation Comment, 1}(2), 12.

Brown, G. T. L. (2004). Teachers' conceptions of assessment: Implications for policy and professional development. \emph{Assessment in Education: Principles, Policy \& Practice, 11}(3), 301--318. \url{https://doi.org/10.1080/0969594042000304609}

Brown, G. T. L. (2006). Teachers' Conceptions of Assessment: Validation of an Abridged Version. \emph{Psychological Reports, 99}(1), 166--170. \url{https://doi.org/10/bf67hf}

DeLuca, C., Willis, J., Cowie, B., Harrison, C., Coombs, A., Gibson, A., \& Trask, S. (2019). Policies, Programs, and Practices: Exploring the Complex Dynamics of Assessment Education in Teacher Education Across Four Countries. \emph{Frontiers in Education, 4}, 132. \url{https://doi.org/10/gh5k2r}

Earl, L. M. (2013). \emph{Assessment as learning: Using classroom assessment to maximize student learning (Second edition)}. Corwin Press.

Fletcher, R. B., Meyer, L. H., Anderson, H., Johnston, P., \& Rees, M. (2012). Faculty and Students Conceptions of Assessment in Higher Education. \emph{Higher Education, 64}(1), 119--133. \url{https://doi.org/10/ctccpq}

Mislevy, R. J. (1994). Test theory reconcieved. \emph{ETS Research Report Series, 1994}(1), i--38. \url{https://doi.org/10/gjm236}

Pellegrino, J. W., Chudowsky, N., \& Glaser, R. (2001). \emph{Knowing What Students Know: The Science and Design of Educational Assessment}. National Academies Press. \url{https://doi.org/10.17226/10019}

Scriven, M. (1967). \emph{The methodology of evaluation.} In B. O. Smith (Ed.), \emph{Perspectives of curriculum evaluation}. Rand McNally

  \bibliography{book.bib}

\end{document}
