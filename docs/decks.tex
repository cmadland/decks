% Options for packages loaded elsewhere
\PassOptionsToPackage{unicode}{hyperref}
\PassOptionsToPackage{hyphens}{url}
%
\documentclass[
]{book}
\usepackage{amsmath,amssymb}
\usepackage{lmodern}
\usepackage{iftex}
\ifPDFTeX
  \usepackage[T1]{fontenc}
  \usepackage[utf8]{inputenc}
  \usepackage{textcomp} % provide euro and other symbols
\else % if luatex or xetex
  \usepackage{unicode-math}
  \defaultfontfeatures{Scale=MatchLowercase}
  \defaultfontfeatures[\rmfamily]{Ligatures=TeX,Scale=1}
\fi
% Use upquote if available, for straight quotes in verbatim environments
\IfFileExists{upquote.sty}{\usepackage{upquote}}{}
\IfFileExists{microtype.sty}{% use microtype if available
  \usepackage[]{microtype}
  \UseMicrotypeSet[protrusion]{basicmath} % disable protrusion for tt fonts
}{}
\makeatletter
\@ifundefined{KOMAClassName}{% if non-KOMA class
  \IfFileExists{parskip.sty}{%
    \usepackage{parskip}
  }{% else
    \setlength{\parindent}{0pt}
    \setlength{\parskip}{6pt plus 2pt minus 1pt}}
}{% if KOMA class
  \KOMAoptions{parskip=half}}
\makeatother
\usepackage{xcolor}
\usepackage{longtable,booktabs,array}
\usepackage{calc} % for calculating minipage widths
% Correct order of tables after \paragraph or \subparagraph
\usepackage{etoolbox}
\makeatletter
\patchcmd\longtable{\par}{\if@noskipsec\mbox{}\fi\par}{}{}
\makeatother
% Allow footnotes in longtable head/foot
\IfFileExists{footnotehyper.sty}{\usepackage{footnotehyper}}{\usepackage{footnote}}
\makesavenoteenv{longtable}
\usepackage{graphicx}
\makeatletter
\def\maxwidth{\ifdim\Gin@nat@width>\linewidth\linewidth\else\Gin@nat@width\fi}
\def\maxheight{\ifdim\Gin@nat@height>\textheight\textheight\else\Gin@nat@height\fi}
\makeatother
% Scale images if necessary, so that they will not overflow the page
% margins by default, and it is still possible to overwrite the defaults
% using explicit options in \includegraphics[width, height, ...]{}
\setkeys{Gin}{width=\maxwidth,height=\maxheight,keepaspectratio}
% Set default figure placement to htbp
\makeatletter
\def\fps@figure{htbp}
\makeatother
\setlength{\emergencystretch}{3em} % prevent overfull lines
\providecommand{\tightlist}{%
  \setlength{\itemsep}{0pt}\setlength{\parskip}{0pt}}
\setcounter{secnumdepth}{5}
\usepackage{booktabs}
\usepackage{amsthm}
\makeatletter
\def\thm@space@setup{%
  \thm@preskip=8pt plus 2pt minus 4pt
  \thm@postskip=\thm@preskip
}
\makeatother

\usepackage{tcolorbox}


\newtcolorbox{protip}{
  colback=black,
  coltext=white,
  colframe=black,
  boxsep=5pt,
  arc=4pt}
\newtcolorbox{bonus}{
  colback=blue!15,
  colframe=blue!15,
  coltext=black!80,
  boxsep=5pt,
  arc=4pt}
\newtcolorbox{reflect}{
  colback=green!5,
  colframe=green!5,
  coltext=black!80,
  boxsep=5pt,
  arc=4pt}
\newtcolorbox{assessment}{
  colback=blue!5,
  colframe=blue!5,
  coltext=black!80,
  boxsep=5pt,
  arc=4pt}
\newtcolorbox{progress}{
  colback=purple!10,
  colframe=purple!10,
  coltext=black!80,
  boxsep=5pt,
  arc=4pt}
\newtcolorbox{video}{
  colback=yellow!5,
  colframe=yellow!5,
  coltext=black!80,
  boxsep=5pt,
  arc=4pt}
\newtcolorbox{caution}{
  colback=red!5,
  colframe=red!5,
  coltext=black!80,
  boxsep=5pt,
  arc=4pt}
\newtcolorbox{feedback}{
  colback=black!5,
  colframe=black!5,
  coltext=black!80,
  boxsep=5pt,
  arc=4pt}
\newtcolorbox{todo}{
  colback=black!5,
  colframe=black!5,
  coltext=black!80,
  boxsep=5pt,
  arc=4pt}
\ifLuaTeX
  \usepackage{selnolig}  % disable illegal ligatures
\fi
\usepackage[]{natbib}
\bibliographystyle{apalike}
\IfFileExists{bookmark.sty}{\usepackage{bookmark}}{\usepackage{hyperref}}
\IfFileExists{xurl.sty}{\usepackage{xurl}}{} % add URL line breaks if available
\urlstyle{same} % disable monospaced font for URLs
\hypersetup{
  pdftitle={Presentations},
  pdfauthor={Colin Madland},
  hidelinks,
  pdfcreator={LaTeX via pandoc}}

\title{Presentations}
\author{Colin Madland}
\date{Last updated: 2023-03-02}

\begin{document}
\maketitle

{
\setcounter{tocdepth}{1}
\tableofcontents
}
\hypertarget{welcome}{%
\chapter*{Welcome}\label{welcome}}
\addcontentsline{toc}{chapter}{Welcome}

Please use the table of contents on the left to navigate through my presentations.

\hypertarget{otessa22---assessment-and-digital-technology-in-higher-education}{%
\chapter{OTESSA22 - Assessment and Digital Technology in Higher Education}\label{otessa22---assessment-and-digital-technology-in-higher-education}}

\hypertarget{introduction}{%
\section*{Introduction}\label{introduction}}
\addcontentsline{toc}{section}{Introduction}

\hypertarget{colin-madland-phd-candidate-university-of-victoria}{%
\subsection*{Colin Madland, PhD Candidate, University of Victoria}\label{colin-madland-phd-candidate-university-of-victoria}}
\addcontentsline{toc}{subsection}{Colin Madland, PhD Candidate, University of Victoria}

Slides - \url{https://bit.ly/otessa22-b}\\
\href{https://cmad.land}{Find me on the web\ldots{}}\\
\href{https://twitter.com/colinmadland}{Twitter}\\
\href{https://scholar.social/web/@Cmadland}{Mastodon}

\textbf{Presented Online at OTESSA22, May 17, 2022}

\begin{quote}
I acknowledge that the land where I currently live and work remains the traditional, ancestral, and unceded land of the \texttt{syilx} (silks) people, whose historical stewardship of and connections to the land continue to today. I am grateful to be an uninvited guest on this land. \href{https://wfn.ca}{To learn more, please visit the Westbank First Nation website.}
\end{quote}

\begin{figure}
\centering
\includegraphics{assets/otessa22/kalamoir.jpg}
\caption{Figure 1. Author's bicycle overlooking Okanagan Lake.}
\end{figure}

\hypertarget{hypothes.is}{%
\subsection*{Hypothes.is}\label{hypothes.is}}
\addcontentsline{toc}{subsection}{Hypothes.is}

\href{https://web.hypothes.is/start/}{If you haven't already, feel free to sign up here as we will use hypothes.is later}. Also, if you have questions or comments, please annotate to your heart's content!

\hypertarget{background}{%
\subsection*{Background}\label{background}}
\addcontentsline{toc}{subsection}{Background}

This review is guided by four research questions:

\begin{enumerate}
\def\labelenumi{\arabic{enumi}.}
\tightlist
\item
  What are the major themes or patterns in the literature related to approaches to assessment in higher education?\\
\item
  What are the major themes or patterns in the literature related to the impact of technology on assessment in higher education?\\
\item
  What gaps exist in the literature related to approaches to assessment in technology-mediated higher education?
\end{enumerate}

\hypertarget{scriven-1967}{%
\subsubsection*{Scriven, 1967}\label{scriven-1967}}
\addcontentsline{toc}{subsubsection}{Scriven, 1967}

\begin{quote}
Scriven, M. (1967). \emph{The methodology of evaluation.} In B. O. Smith (Ed.), \emph{Perspectives of curriculum evaluation}. Rand McNally
\end{quote}

\begin{itemize}
\tightlist
\item
  distinction between \texttt{formative} and \texttt{summative}
\end{itemize}

\hypertarget{bloom-1968}{%
\subsubsection*{Bloom, 1968}\label{bloom-1968}}
\addcontentsline{toc}{subsubsection}{Bloom, 1968}

\begin{quote}
Bloom, B. (1968). Learning for Mastery. Instruction and Curriculum. Regional Education Laboratory for the Carolinas and Virginia, Topical Papers and Reprints, Number 1. \emph{Evaluation Comment, 1}(2), 12.
\end{quote}

\begin{itemize}
\tightlist
\item
  Incorporated \texttt{formative} and \texttt{summative} distinction into his ideas about \texttt{mastery\ learning}
\end{itemize}

\hypertarget{mislevy-1994}{%
\subsubsection*{Mislevy, 1994}\label{mislevy-1994}}
\addcontentsline{toc}{subsubsection}{Mislevy, 1994}

\begin{quote}
Mislevy, R. J. (1994). Test theory reconcieved. \emph{ETS Research Report Series, 1994}(1), i--38. \url{https://doi.org/10/gjm236}
\end{quote}

\begin{itemize}
\item
  \begin{quote}
  test theory is machinery for reasoning from students' behavior to conjectures about their competence, as framed in a particular conception of competence.''(p.~4).
  \end{quote}
\end{itemize}

\hypertarget{black-and-wiliam-1998}{%
\subsubsection*{Black and Wiliam, 1998}\label{black-and-wiliam-1998}}
\addcontentsline{toc}{subsubsection}{Black and Wiliam, 1998}

\begin{quote}
Black, P., \& Wiliam, D. (1998). Assessment and Classroom Learning. \emph{Assessment in Education: Principles, Policy \& Practice, 5}(1), 7--74. \url{https://doi.org/10/fpnss4}
\end{quote}

\begin{itemize}
\item
  major review of the literature on \texttt{formative\ assessment}\\
\item
  describe formative assessment as encouraging gains in achievement that were

\begin{verbatim}
> among the largest ever reported for educational interventions (p. 61)
\end{verbatim}
\end{itemize}

\hypertarget{pellegrino-et-al.-2001}{%
\subsubsection*{Pellegrino et al., 2001}\label{pellegrino-et-al.-2001}}
\addcontentsline{toc}{subsubsection}{Pellegrino et al., 2001}

\begin{quote}
Pellegrino, J. W., Chudowsky, N., \& Glaser, R. (2001). \emph{Knowing What Students Know: The Science and Design of Educational Assessment}. National Academies Press. \url{https://doi.org/10.17226/10019}
\end{quote}

\begin{itemize}
\tightlist
\item
  ``a process of drawing reasonable inferences about what students know on the basis of evidence derived from observations of what they say, do, or make in selected situations'' (p.~112)\\
\item
  ``reasoning from evidence'' (p.~43)
\end{itemize}

\hypertarget{assessment-triangle}{%
\paragraph*{Assessment Triangle}\label{assessment-triangle}}
\addcontentsline{toc}{paragraph}{Assessment Triangle}

\begin{figure}
\centering
\includegraphics{assets/otessa22/assessment-triangle.png}
\caption{Figure 2. Assessment Triangle from Pellegrino et al.~(2001)}
\end{figure}

\hypertarget{cognition}{%
\subparagraph*{Cognition}\label{cognition}}
\addcontentsline{toc}{subparagraph}{Cognition}

\begin{itemize}
\tightlist
\item
  a cognitive model of the domain
\end{itemize}

\hypertarget{observation}{%
\subparagraph*{Observation}\label{observation}}
\addcontentsline{toc}{subparagraph}{Observation}

\begin{itemize}
\tightlist
\item
  a performance task used to gather data regarding learner achievement
\end{itemize}

\hypertarget{interpretation}{%
\subparagraph*{Interpretation}\label{interpretation}}
\addcontentsline{toc}{subparagraph}{Interpretation}

\begin{itemize}
\tightlist
\item
  an inference or judgement of the learner's achievement in relation to the model of the domain
\end{itemize}

\hypertarget{approaches-to-learning}{%
\subsection*{Approaches to Learning}\label{approaches-to-learning}}
\addcontentsline{toc}{subsection}{Approaches to Learning}

\hypertarget{biggs-1993}{%
\subsubsection*{Biggs, 1993}\label{biggs-1993}}
\addcontentsline{toc}{subsubsection}{Biggs, 1993}

\begin{quote}
Biggs, J. B. (1993). From Theory to Practice: A Cognitive Systems Approach. \emph{Higher Education Research \& Development, 12}(1), 73--85. \url{https://doi.org/10/ccdmd9}
\end{quote}

\begin{figure}
\centering
\includegraphics{assets/otessa22/3p-model.png}
\caption{Figure 3. 3-P Model of Teaching and Learning adapted from Biggs (1993)}
\end{figure}

\hypertarget{presage}{%
\paragraph*{Presage}\label{presage}}
\addcontentsline{toc}{paragraph}{Presage}

\begin{itemize}
\tightlist
\item
  factors that precede learning activities

  \begin{itemize}
  \tightlist
  \item
    learner factors

    \begin{itemize}
    \tightlist
    \item
      prior knowledge\\
    \item
      educational experience\\
    \item
      affective states\\
    \item
      wellness (physical \& mental)\\
    \end{itemize}
  \item
    teacher factors

    \begin{itemize}
    \tightlist
    \item
      vertical \& horizontal discourses (Bernstein, 1999)\\
    \item
      institutional policies\\
    \item
      department norms\\
    \item
      educational experiences
    \end{itemize}
  \end{itemize}
\end{itemize}

\hypertarget{process}{%
\paragraph*{Process}\label{process}}
\addcontentsline{toc}{paragraph}{Process}

\begin{itemize}
\tightlist
\item
  learning focused activities

  \begin{itemize}
  \tightlist
  \item
    reading, writing, discussing, building, creating, synthesizing, researching, sharing, debating, publishing\ldots{}
  \end{itemize}
\item
  surface approaches

  \begin{itemize}
  \tightlist
  \item
    using low-level cognitive skills when high-level cognitive skills are required
  \end{itemize}
\item
  deep approaches

  \begin{itemize}
  \tightlist
  \item
    using high-level cognitive skills for tasks which require them
  \end{itemize}
\end{itemize}

\hypertarget{product}{%
\paragraph*{Product}\label{product}}
\addcontentsline{toc}{paragraph}{Product}

\begin{itemize}
\tightlist
\item
  learner achievement of outcomes (intended or emergent)
\item
  fed back into the system

  \begin{itemize}
  \tightlist
  \item
    informs learners and instructors
  \end{itemize}
\end{itemize}

\hypertarget{conceptions-of-assessment}{%
\subsection*{Conceptions of Assessment}\label{conceptions-of-assessment}}
\addcontentsline{toc}{subsection}{Conceptions of Assessment}

\hypertarget{brown-1994-1996}{%
\subsubsection*{Brown, 1994; 1996}\label{brown-1994-1996}}
\addcontentsline{toc}{subsubsection}{Brown, 1994; 1996}

\begin{quote}
Brown, G. T. L. (2004). Teachers' conceptions of assessment: Implications for policy and professional development. \emph{Assessment in Education: Principles, Policy \& Practice, 11}(3), 301--318. \url{https://doi.org/10.1080/0969594042000304609}
\end{quote}

\begin{quote}
Brown, G. T. L. (2006). Teachers' Conceptions of Assessment: Validation of an Abridged Version. \emph{Psychological Reports, 99}(1), 166--170. \url{https://doi.org/10/bf67hf}
\end{quote}

\begin{itemize}
\tightlist
\item
  general mental structure, encompassing beliefs, meanings, concepts, propositions, rules, mental images, preferences

  \begin{itemize}
  \tightlist
  \item
    improvement of teaching and learning,\\
  \item
    school accountability,\\
  \item
    student accountability, or\\
  \item
    treating assessment as irrelevant.
  \end{itemize}
\end{itemize}

\hypertarget{fletcher-et-al.-2012}{%
\subsubsection*{Fletcher et al., 2012}\label{fletcher-et-al.-2012}}
\addcontentsline{toc}{subsubsection}{Fletcher et al., 2012}

\begin{quote}
Fletcher, R. B., Meyer, L. H., Anderson, H., Johnston, P., \& Rees, M. (2012). Faculty and Students Conceptions of Assessment in Higher Education. \emph{Higher Education, 64}(1), 119--133. \url{https://doi.org/10/ctccpq}
\end{quote}

\begin{itemize}
\tightlist
\item
  instructors were more likely than learners to view assessment as consistent and trustworthy methods to understand and improve learning\\
\item
  learners were more likely to have negative views of assessment and viewed it as a measure of student and institutional accountability.
\end{itemize}

\hypertarget{earl-2013}{%
\subsubsection*{Earl, 2013}\label{earl-2013}}
\addcontentsline{toc}{subsubsection}{Earl, 2013}

\begin{quote}
Earl, L. M. (2013). \emph{Assessment as learning: Using classroom assessment to maximize student learning (Second edition)}. Corwin Press.
\end{quote}

\begin{itemize}
\tightlist
\item
  Assessment \emph{OF} Learning

  \begin{itemize}
  \tightlist
  \item
    summative
  \end{itemize}
\item
  Assessment \emph{FOR} Learning

  \begin{itemize}
  \tightlist
  \item
    formative
  \end{itemize}
\item
  Assessment \emph{AS} Learning

  \begin{itemize}
  \tightlist
  \item
    metacognitive
  \end{itemize}
\end{itemize}

\hypertarget{approaches-to-assessment}{%
\subsection*{Approaches to Assessment}\label{approaches-to-assessment}}
\addcontentsline{toc}{subsection}{Approaches to Assessment}

Both learning and assessment are complex phenomena which are impacted by myriad factors.

\hypertarget{shepard-2000}{%
\subsubsection*{Shepard (2000)}\label{shepard-2000}}
\addcontentsline{toc}{subsubsection}{Shepard (2000)}

\begin{quote}
Shepard, L. A. (2000). The Role of Assessment in a Learning Culture. \emph{Educational Researcher, 29}(7), 4--14. \url{https://doi.org/10/cw9jwc}
\end{quote}

\begin{itemize}
\tightlist
\item
  traditional assessment structures originated in behaviourist models of teaching and learning

  \begin{itemize}
  \tightlist
  \item
    emphasis on culture of summative assessment
  \end{itemize}
\item
  modern constructivist models of teaching and learning are less compatible with previous assessment structures, yet a culture that emphasizes summative assessment seems to persist alongside emerging models of assessment
\end{itemize}

\hypertarget{deluca-2016}{%
\subsubsection*{DeLuca, 2016}\label{deluca-2016}}
\addcontentsline{toc}{subsubsection}{DeLuca, 2016}

\begin{quote}
DeLuca, C., LaPointe-McEwan, D., \& Luhanga, U. (2016). Approaches to classroom assessment inventory: A new instrument to support teacher assessment literacy. \emph{Educational Assessment, 21}, 248--266. \url{https://doi.org/10/gfgtsg}
\end{quote}

\begin{figure}
\centering
\includegraphics{assets/otessa22/approaches.png}
\caption{Figure 4. Approaches to Classroom Assessment from DeLuca et al.~(2016)}
\end{figure}

\begin{itemize}
\tightlist
\item
  \emph{Approaches to Classroom Assessment Inventory}

  \begin{itemize}
  \tightlist
  \item
    designed to inventory K12 teachers' thoughts, beliefs, actions related to assessment

    \begin{itemize}
    \tightlist
    \item
      Assessment purpose (of, for, as learning)\\
    \item
      Assessment process (design, use/scoring, communication)\\
    \item
      Assessment fairness (standard, equitable, differentiated)\\
    \item
      Assessment theory (consistent, balanced, contextual)
    \end{itemize}
  \end{itemize}
\end{itemize}

\hypertarget{technology-mediated-assessment-in-higher-education}{%
\section*{Technology-Mediated Assessment in Higher Education}\label{technology-mediated-assessment-in-higher-education}}
\addcontentsline{toc}{section}{Technology-Mediated Assessment in Higher Education}

\hypertarget{contrasting-with-k12}{%
\subsection*{Contrasting with K12}\label{contrasting-with-k12}}
\addcontentsline{toc}{subsection}{Contrasting with K12}

There is a very large body of literature on assessment in K12 learning contexts, and a not-quite as large, but still substantial body of literature on assessment in higher education. It may be tempting to conflate the two contexts, but K12 teachers typically complete 2 full years of pedagogical training as part of their academic and practical preparation. These two years often include specific courses on assessment, learning theory, as well as domain-specific pedagogies.

On the other hand, higher education instructors (from part-time sessionals to adjuncts to tenure-track and tenured faculty) tend to engage in little academic preparation in learning theories or assessment, although they seem to absorb the signature pedagogies of their discipline.

\hypertarget{impact-of-technology}{%
\subsection*{Impact of Technology}\label{impact-of-technology}}
\addcontentsline{toc}{subsection}{Impact of Technology}

\begin{itemize}
\tightlist
\item
  Impact on higher education is ubiquitous (SIS, LMS/VLE, CRM, etc.)
\item
  Tends to emphasize \texttt{efficiency} (however ill-defined that may be)

  \begin{itemize}
  \tightlist
  \item
    doing the same things with greater speed and/or reduced effort
  \item
    reinscribes mis-aligned assessment structures
  \end{itemize}
\end{itemize}

\hypertarget{pockets-of-innovation}{%
\subsubsection*{Pockets of Innovation}\label{pockets-of-innovation}}
\addcontentsline{toc}{subsubsection}{Pockets of Innovation}

\hypertarget{bearman-et-al.-2020}{%
\subsubsection*{Bearman et al.~2020}\label{bearman-et-al.-2020}}
\addcontentsline{toc}{subsubsection}{Bearman et al.~2020}

\begin{quote}
Bearman, M., Dawson, P., Ajjawi, R., Tai, J., \& Boud, D. (Eds.). (2020). \emph{Re-imagining university assessment in a digital world.} Springer.
\end{quote}

\begin{itemize}
\tightlist
\item
  cognitive offloading\\
\item
  artificial intelligence

  \begin{itemize}
  \tightlist
  \item
    ``personalized'' learning; recommender systems, automated item generation, automated essay scoring\\
  \end{itemize}
\item
  dialogic feedback

  \begin{itemize}
  \tightlist
  \item
    video, audio, screencast\\
  \end{itemize}
\item
  data \& learning analytics

  \begin{itemize}
  \tightlist
  \item
    process data\\
  \end{itemize}
\item
  peer/self-assessment
\item
  micro-credentials
\end{itemize}

However\ldots{}

\begin{itemize}
\tightlist
\item
  critical to consider ethical and social impacts!

  \begin{itemize}
  \tightlist
  \item
    surveillance
  \item
    equity
  \item
    algorithmic assessment
  \end{itemize}
\end{itemize}

\hypertarget{bower-2019}{%
\subsubsection*{Bower, 2019}\label{bower-2019}}
\addcontentsline{toc}{subsubsection}{Bower, 2019}

\begin{quote}
Bower, M. (2019). Technology‐mediated learning theory. \emph{British Journal of Educational Technology, 50}(3), 1035--1048. \url{https://doi.org/10.1111/bjet.12771}
\end{quote}

\begin{quote}
In technology-mediated learning contexts, agentic intentions reside with humans, and not with technology.
\end{quote}

\begin{itemize}
\tightlist
\item
  3 (select) premises

  \begin{itemize}
  \tightlist
  \item
    technology \texttt{mediates} between learners and outcomes\\
  \item
    beliefs, knowledge, practices, and environment are mutually influential (add this to the complexity of assessment)\\
  \item
    role of teachers is to optimise learning through the \texttt{purposeful\ deployment} of learning technologies
  \end{itemize}
\end{itemize}

\hypertarget{revisiting-shepard-2000}{%
\subsection*{Revisiting Shepard (2000)}\label{revisiting-shepard-2000}}
\addcontentsline{toc}{subsection}{Revisiting Shepard (2000)}

\textbf{Using hypothes.is}

\begin{itemize}
\tightlist
\item
  22 years have passed\ldots{}\\
\item
  What has changed?\\
\item
  What is your experience of technology-mediated assessment in higher education?\\
\item
  What are your greatest challenges related to technology-mediated assessment?
\end{itemize}

\hypertarget{themes-and-research-directions}{%
\section*{Themes and Research Directions}\label{themes-and-research-directions}}
\addcontentsline{toc}{section}{Themes and Research Directions}

\begin{itemize}
\tightlist
\item
  assessment as conversation in digital environments\\
\item
  validity exploration of \emph{Approaches to Assessment} in higher ed.\\
\item
  humanizing assessment, ethics
\end{itemize}

\hypertarget{questions-comments}{%
\section*{Questions? Comments?}\label{questions-comments}}
\addcontentsline{toc}{section}{Questions? Comments?}

\hypertarget{references}{%
\section*{References}\label{references}}
\addcontentsline{toc}{section}{References}

Bearman, M., Dawson, P., Ajjawi, R., Tai, J., \& Boud, D. (Eds.). (2020). \emph{Re-imagining university assessment in a digital world.} Springer.

Bernstein, B. (1999). Vertical and Horizontal Discourse: An Essay. \emph{British Journal of Sociology of Education, 20}(2), 157--173. JSTOR. \url{https://doi.org/10/ftmsvc}

Biggs, J. B. (1993). From Theory to Practice: A Cognitive Systems Approach. \emph{Higher Education Research \& Development, 12}(1), 73--85. \url{https://doi.org/10/ccdmd9}

Black, P., \& Wiliam, D. (1998). Assessment and Classroom Learning. \emph{Assessment in Education: Principles, Policy \& Practice, 5}(1), 7--74. \url{https://doi.org/10/fpnss4}

Bloom, B. (1968). Learning for Mastery. Instruction and Curriculum. Regional Education Laboratory for the Carolinas and Virginia, Topical Papers and Reprints, Number 1. \emph{Evaluation Comment, 1}(2), 12.

Bower, M. (2019). Technology‐mediated learning theory. \emph{British Journal of Educational Technology, 50}(3), 1035--1048. \url{https://doi.org/10.1111/bjet.12771}

Brown, G. T. L. (2004). Teachers' conceptions of assessment: Implications for policy and professional development. \emph{Assessment in Education: Principles, Policy \& Practice, 11}(3), 301--318. \url{https://doi.org/10.1080/0969594042000304609}

Brown, G. T. L. (2006). Teachers' Conceptions of Assessment: Validation of an Abridged Version. \emph{Psychological Reports, 99}(1), 166--170. \url{https://doi.org/10/bf67hf}

DeLuca, C., LaPointe-McEwan, D., \& Luhanga, U. (2016). Approaches to classroom assessment inventory: A new instrument to support teacher assessment literacy. \emph{Educational Assessment, 21}, 248--266. \url{https://doi.org/10/gfgtsg}

DeLuca, C., Willis, J., Cowie, B., Harrison, C., Coombs, A., Gibson, A., \& Trask, S. (2019). Policies, Programs, and Practices: Exploring the Complex Dynamics of Assessment Education in Teacher Education Across Four Countries. \emph{Frontiers in Education, 4}, 132. \url{https://doi.org/10/gh5k2r}

Earl, L. M. (2013). \emph{Assessment as learning: Using classroom assessment to maximize student learning (Second edition)}. Corwin Press.

Fletcher, R. B., Meyer, L. H., Anderson, H., Johnston, P., \& Rees, M. (2012). Faculty and Students Conceptions of Assessment in Higher Education. \emph{Higher Education, 64}(1), 119--133. \url{https://doi.org/10/ctccpq}

Mislevy, R. J. (1994). Test theory reconcieved. \emph{ETS Research Report Series, 1994}(1), i--38. \url{https://doi.org/10/gjm236}

Pellegrino, J. W., Chudowsky, N., \& Glaser, R. (2001). \emph{Knowing What Students Know: The Science and Design of Educational Assessment}. National Academies Press. \url{https://doi.org/10.17226/10019}

Scriven, M. (1967). \emph{The methodology of evaluation.} In B. O. Smith (Ed.), \emph{Perspectives of curriculum evaluation}. Rand McNally

Shepard, L. A. (2000). The Role of Assessment in a Learning Culture. \emph{Educational Researcher, 29}(7), 4--14. \url{https://doi.org/10/cw9jwc}

\hypertarget{twu-faculty-professional-learning}{%
\chapter{TWU Faculty Professional Learning}\label{twu-faculty-professional-learning}}

\hypertarget{colin-madland-manager-online-learning-and-instructional-technology-twu-global}{%
\subsection*{Colin Madland, Manager, Online Learning and Instructional Technology (TWU GLOBAL)}\label{colin-madland-manager-online-learning-and-instructional-technology-twu-global}}
\addcontentsline{toc}{subsection}{Colin Madland, Manager, Online Learning and Instructional Technology (TWU GLOBAL)}

PhD Candidate, University of Victoria

Notes - \url{https://bit.ly/twu-assessment}

\begin{figure}
\centering
\includegraphics{assets/twu-asmt/twu-asmt.png}
\caption{QR Code to access presentation notes}
\end{figure}

\href{https://cmad.land}{Find me on the web\ldots{}}\\
\href{https://twitter.com/colinmadland}{Twitter}\\
\href{https://scholar.social/web/@Cmadland}{Mastodon}

\textbf{Presented Online for TWU Faculty Professional Learning, Thursday, March 9, 2023}

\begin{quote}
I acknowledge that the land where I currently live and work remains the traditional, ancestral, and unceded land of the \texttt{syilx} (silks) people, whose historical stewardship of and connections to the land continue to today. I am grateful to be an uninvited guest on this land. \href{https:/syilx.org}{To learn more, please visit the syilx.org.}
\end{quote}

\begin{figure}
\centering
\includegraphics{assets/twu-asmt/ellie.jpeg}
\caption{My blind dog, Eleanor near the top of Mission Hill, overlooking syilx territory.}
\end{figure}

\hypertarget{what-is-assessment}{%
\section{What is `assessment'?}\label{what-is-assessment}}

\begin{quote}
Pellegrino, J. W., Chudowsky, N., \& Glaser, R. \emph{Knowing What Students Know: The Science and Design of Educational Assessment}. National Academies Press. \url{https://doi.org/10.17226/10019}
\end{quote}

\begin{itemize}
\tightlist
\item
  ``reasoning from evidence'' (p.~43)\\
\item
  ``a \emph{process} of drawing reasonable \emph{inferences} about what students know on the basis of \emph{evidence} derived from \emph{observations} of what they say, do, or make in \emph{selected situations}'' (p.~112)
\end{itemize}

\hypertarget{thinking-about-your-approach-to-assessment}{%
\section{Thinking about your Approach to Assessment}\label{thinking-about-your-approach-to-assessment}}

Scenario 1

\begin{quote}
You are teaching a large enrollment course. The students will be submitting bi-weekly assignments, a midterm exam, and a culminating assignment all designed to support their learning.
\end{quote}

\begin{itemize}
\tightlist
\item
  Grade each bi-weekly assignment.\\
\item
  Read a subset of bi-weekly assignments, identify and share performance trends with the whole class.\\
\item
  Ask students to self-assess their bi-weekly assignments using evaluative criteria.\\
\item
  Use performance trends from the bi-weekly assignments to inform the redesign of the midterm or culminating assessment.\\
\item
  Develop a rubric or scoring guide to assess the culminating assignment in advance of student submissions.\\
\item
  Dedicate class time to discuss students' performance trends from the midterm and address gaps in learning so students are better prepared for the culminating assignment.\\
\item
  Have every student complete the same culminating assignment using the same scoring rubric or guide.\\
\item
  Have all students complete the same culminating assignment with formal accommodations for students who require them.\\
\item
  Provide students with a choice of three different culminating assignments that assess the same learning goals.\\
\item
  Consistently apply late submission policies for all students when generating grades.\\
\item
  Consider each student's individual circumstances when deciding how to apply late submission policies.\\
\item
  Use the late submission policy as a guideline to ensure a consistent principle is applied while also using professional judgement for students with individual circumstances.
\end{itemize}

If your response to this scenario was not listed above, how would you most likely respond?

\begin{itemize}
\item
\end{itemize}

Scenario 2

\begin{quote}
A core assignment in your course involves students working in groups online.
\end{quote}

\begin{itemize}
\tightlist
\item
  Grade the assignment solely based on the group's final product.
\item
  Engage students in an ongoing peer feedback process to enhance group collaboration.
\item
  Engage students in self-assessment to increase their accountability and engagement in the assignment.
\item
  Leverage online design features to engage students in peer feedback and self-assessment.
\item
  Take group member feedback into consideration when generating final grades.
\item
  Communicate grading decisions based on evaluation criteria and evidence of student learning.
\item
  Give all group members the same grade.
\item
  Monitor barriers to a student's performance in group work (e.g., language, technology) and grade accordingly.
\item
  Grade each student individually based on their contribution to the group's process and product.
\item
  Use the same rubric to consistently grade all groups' assignments.
\item
  Modify and apply the rubric differently in response to unexpected group events (e.g., group member leaves).
\item
  Use the same rubric, but consider group composition, size, and cohesion when grading.
\end{itemize}

If your response to this scenario was not listed above, how would you most likely respond?

Scenario 3

\begin{quote}
There are expectations in your department that grades should be distributed across the grading scale. However, your class averages are consistently lower than your colleagues'. Your course assessment scheme includes two term exams and one final exam.
\end{quote}

\begin{itemize}
\tightlist
\item
  Provide students with additional graded assessments to chunk learning into smaller units.
\item
  Provide students with additional opportunities to check their understanding throughout the course (e.g., ungraded quizzes, exit slips).
\item
  Provide self-assessment opportunities to help students recognize and address gaps in their learning.
\item
  Analyze exam results to determine if weak performance was due to exam design issues.
\item
  Remove exam questions that most students struggled with and re-calculate student scores.
\item
  Schedule class time to review exam performance to address learning gaps.
\item
  Shift all exam grades up so averages are consistent with departmental colleagues.
\item
  Provide students who performed below the class average with the opportunity to rewrite an equivalent exam.
\item
  Provide any student the opportunity to rewrite an equivalent exam.
\item
  Analyze the consistency of student performance across course exams.
\item
  Analyze exam questions to ensure alignment with taught content.
\item
  Analyze how students performed on exams in relation to taught content.
\end{itemize}

If your response to this scenario was not listed above, how would you most likely respond?

Scenario 4

\begin{quote}
You teach a course with multiple sections taught by various instructors. Your students have complained to you that assignments are constructed and graded differently across sections.
\end{quote}

\begin{itemize}
\tightlist
\item
  Explain to students that while learning outcomes are the same across sections, grades are based on their individual performance on assignments.
\item
  Ensure students have the opportunity to receive feedback prior to submitting assignments to increase their focus on learning over grades.
\item
  Invite students to reflect on their personal learning goals for the course so they can plan for their own success in your course.
\item
  Engage in a collaborative design process with other instructors to set standards and design common assignments.
\item
  Collaboratively score a subset of assignments with the other instructors to ensure consistent use of scoring guides across sections.
\item
  Communicate your grading approach to your students and explain how it aligns with intended learning goals.
\item
  Propose a standard approach to assignments and grading be applied across all sections.
\item
  Assure students that while the assignments may be different across the sections, they assess the same learning outcomes.
\item
  Offer students the opportunity to select and complete an assignment from another section.
\item
  Work with the other instructors to revise all assignments across sections so they are all the same.
\item
  Justify differences in approaches to assessment based on instructor orientations to assessment, teaching context, and students' learning needs.
\item
  Recognize student concerns and engage in practices that ensure your assignments are equivalent to those in other sections.
\end{itemize}

If your response to this scenario was not listed above, how would you most likely respond?

Scenario 5

\begin{quote}
You discover that a student has plagiarized some of their assignment (e.g., an essay, lab report).
\end{quote}

\begin{itemize}
\tightlist
\item
  Give the student a 0 on the assignment.
\item
  Have the student re-write the plagiarized section in their own words, then re-grade the assignment.
\item
  Ask the students to reflect on why plagiarism is a problem and what they would do differently next time.
\item
  As the instructor, reflect on how the assignment could have been structured differently to deter plagiarism.
\item
  Adjust the student's grade to reflect the portion of work that was plagiarized.
\item
  Discuss with the student the reasons for the plagiarism, severity of plagiarism, and negotiate potential next steps for their learning.
\item
  Apply the same consequence you would for other students to ensure all students are treated the same.
\item
  Consider if the student has identified accommodations before determining response to plagiarism.
\item
  Discuss why the student plagiarized and agree upon an appropriate alternative assignment.
\item
  Apply all aspects of institutional policy on academic integrity to ensure consistency across all students.
\item
  Consider the original aspects of the assignment and the plagiarized text to determine what the student knows and does not appear to know related to learning outcomes.
\item
  Consider extenuating circumstances surrounding the plagiarism and use professional judgement when applying the institutional academic integrity policy.
\end{itemize}

If your response to this scenario was not listed above, how would you most likely respond?

\begin{reflect}
The Scenarios and responses above are from DeLuca, C., LaPointe-McEwan,
D., \& Luhanga, U. (2016). \href{https://doi.org/10/gfgtsg}{Approaches
to classroom assessment inventory: A new instrument to support teacher
assessment literacy.} \emph{Educational Assessment, 21}, 248--266.
\end{reflect}

  \bibliography{book.bib}

\end{document}
